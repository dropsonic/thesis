\documentclass[oneside,final,14pt]{extreport}
\usepackage{pdfpages} % поддержка вставки страниц из pdf-файлов
\usepackage[onehalfspacing]{setspace} % 1,5 интервал
\usepackage[top=2.0cm,bottom=2.0cm,left=2.0cm,right=1.0cm]{geometry} % поля
\usepackage{lscape} % поддержка альбомной ориентации страниц
\usepackage{indentfirst} % красная строка
\setlength\parindent{1.5cm} % установка величины отступа красной строки

\usepackage{etoolbox} % позволяет добавлять произвольный код в начало любой команды, и содержит прочие подобные доп. функции для программирования
\usepackage{suffix} % позволяет легко определять команды со звёздочкой (starred version of command)

% Шрифты
\usepackage[cm-default]{fontspec}
\usepackage{xunicode}
\usepackage{xltxtra}
%\setromanfont[Mapping=tex-text]{Times New Roman}
\setmainfont[Mapping=tex-text]{Times New Roman}
\setsansfont{Calibri}
\setmonofont{Consolas}
\defaultfontfeatures{Scale=MatchLowercase, Mapping=tex-text} % одинаковый рост строчных букв у разных гарнитур, маппинги TeXовских лигатур вроде -- и ---
\usepackage{ulem} % поддержка подчёркиваний

% Поддержка русского языка и русскоязычных стилей
\usepackage{polyglossia}
\setmainlanguage[babelshorthands=true]{russian} % основной язык - русский
\setotherlanguage[variant=us]{english} % дополнительный язык - английский

%\usepackage{amsmath}
% Формулы
\usepackage{amstext} % поддержка кириллицы в математических формулах
\usepackage{amssymb} % дополнительные символы в математических формулах
\usepackage{chngcntr} % управление нумерацией
\counterwithout{equation}{chapter} % сквозная нумерация формул

% Формат заголовков
\usepackage{titlesec}
\newcommand{\nhspacesize}{10pt}
\newcommand{\nohyphenation}{\righthyphenmin62}
\titleformat{\chapter}[hang]{\nohyphenation\sloppy\Large\bfseries}{\thechapter}{\nhspacesize}{} % righthypenmin62 - не переносить слова в названиях глав
\titleformat{\section}[hang]{\nohyphenation\sloppy\large\bfseries}{\thesection}{\nhspacesize}{}
\titleformat{\subsection}[hang]{\sloppy\normalsize\bfseries}{\thesubsection}{\nhspacesize}{}
\titlespacing*{\chapter}{\parindent}{-30pt}{*4}
\titlespacing*{\section}{\parindent}{*1}{*2}
\titlespacing*{\subsection}{\parindent}{*1}{*1}
% Новый формат заголовков - специальные разделы (реферат, вступление, заключение и т.п.)
\newcommand{\spchapterStar}[1]{
	\clearpage
	\begin{center}
		\nohyphenation{\sloppy{\textbf{\Large{#1}}}}
	\end{center}
	\par}
\newcommand{\spchapter}[1]{
	\spchapterStar{#1}
	\addcontentsline{toc}{chapter}{#1}}
\WithSuffix\newcommand\spchapter*[1]{\spchapterStar{#1}}

% Формат списков
\usepackage{enumitem}
\setlist{nolistsep} % убрать лишний интервал между элементами списка
\setlist[itemize,1]{label={–}, labelindent=\parindent, leftmargin=*} % маркированные списки: символ - короткое тире, выравнивание символа по красной строке
\AddEnumerateCounter{\asbuk}{\asbuk}{д} % последний параметр - самый широкий символ в перечислении
\setlist[enumerate,1]{label={\asbuk*}), labelindent=\parindent, leftmargin=*}
\setlist[enumerate,2]{label={\arabic*}), leftmargin=\parindent}

% Поддержка изображений
\usepackage{graphicx}
\graphicspath{{./images/chapter1/}{./images/chapter2/}{./images/chapter3/}{./images/chapter4/}} % пути к каталогам с изображениями
\usepackage{svg} % поддержка вставки векторных (SVG) изображений (из Inkscape) - команда includesvg

% Красивые таблицы
\usepackage{makecell}
\usepackage{multirow}

% Формат рисунков и таблиц
\usepackage{caption} % подписи
\captionsetup{labelsep=endash, textformat=simple, figurename=Рисунок, tablename=Таблица, figurewithin=none, tablewithin=none} % разделитель подписи и названия - короткое тире, заголовок - название float'a и номер, имена рисунков и таблиц по ГОСТу, сквозная нумерация рисунков и таблиц
\captionsetup[figure]{position=above}
\captionsetup[table]{singlelinecheck=false, position=top, justification=raggedright}
% подписи многостраничных таблиц
\DeclareCaptionLabelFormat{continued}{#1~#2 (\textit{продолжение})}
\captionsetup[ContinuedFloat]{labelformat=continued}
\usepackage{floatrow} % пакет для настройки размещения float'ов и их подписей
\floatsetup[table]{style=plaintop, justification=justified} % название над таблицей, таблица выровнена по ширине влево

% Библиография и библиографические ссылки
\usepackage{cite}
% Замена формата нумерации списка литературы с "[1]" на "1."
\makeatletter
\renewcommand{\@biblabel}[1]{#1.}
\makeatother
\bibliographystyle{utf8gost705u}
\gappto\captionsrussian{\renewcommand{\bibname}{Список использованных источников}}

% Оглавление
\gappto\captionsrussian{\renewcommand{\contentsname}{Содержание}}
\usepackage[subfigure,titles]{tocloft}
\renewcommand{\cftchapleader}{\bfseries\cftdotfill{\cftdotsep}}

% Подсчёт объектов (для реферата)
\usepackage{totcount}
\regtotcounter{figure} % рисунки
\regtotcounter{table} % таблицы
% подсчёт приложений
\newtotcounter{appendixcount}
\usepackage{apptools}
\pretocmd{\chapter}{\IfAppendix{\addtocounter{appendixcount}{1}}{}}{}{}
% подсчёт страниц
\usepackage{lastpage}
\newcommand{\pagecount}{\pageref{LastPage}}
% подсчёт использованных источников
\newtotcounter{refcount}
\pretocmd{\bibitem}{\addtocounter{refcount}{1}}{}{}

% Оформление приложений
\usepackage[title, titletoc]{appendix}
% задание своего формата заголовков для приложений
\pretocmd{\appendix}{
	\titleformat{\chapter}[display]{\nohyphenation\sloppy\normalsize\centering}{\MakeUppercase{\chaptertitlename} \thechapter}{\nhspacesize}{\bfseries}{}
}{}{}

\begin{document}
	\includepdf[pages={1}]{title.pdf} % титульник
	\includepdf[pages={1,2}]{assignment.pdf} % задание
	\setcounter{page}{3} % начать нумерацию страниц с №3
	\include{referat}
	\tableofcontents
	
	\spchapter{Введение}
*здесь должно быть введение*
	\include{chapter1}
	\chapter{Расчет экономической эффективности системы}
\section{Введение}
Для оценки экономической эффективности программно-аппаратного продукта требуется:
\begin{itemize}
\item определить целесообразность разработки;
\item определить трудоёмкость и затраты на создание;
\item определить показатели экономической эффективности разработки.
\end{itemize}

Результатом выполнения данной части является обоснование технической, экономической и научной значимости и целесообразности продукта. Объектом технико-экономического анализа является программная система мониторинга состояния ЛА на основе методов интеллектуального анализа данных.

\section{Определение целесообразности разработки}
Для обоснования целесообразности разработки продукта необходимо:
\begin{itemize}
\item выбрать аналог (если таковой имеется);
\item сформулировать перечень функциональных характеристик по предлагаемому варианту разработки продукта;
\item определить конкретные уровни характеристик и их значимость;
\item определить индекс технического уровня программного продукта.
\end{itemize}

Функционально-технические характеристики разрабатываемого программного продукта представлены в таблице~\ref{tab:econom:characteristics}.

\begin{table}
\caption{Функционально-технические характеристики}
\nohyphenation
\label{tab:econom:characteristics}
%\begin{tabularx}{\textwidth}{|X|X|X|X|X|}
\begin{tabular}{|m{\ptw{0.265}}|m{\ptw{0.179}}|m{\ptw{0.159}}|m{\ptw{0.159}}|m{\ptw{0.238}}|}
%\begin{tabular}{|m{4cm}|m{2.7cm}|m{2.4cm}|m{2.4cm}|m{3.6cm}|}
\hline \multirow{2}{\hsize}{Функциональные характеристики} & \multirow{2}{\hsize}{Единица измерения} & \multicolumn{2}{m{\ptw{0.318}}|}{Величина функциональных характеристик} & \multirow{2}{\hsize}{Значимость характеристик} \\
\cline{3-4} & & Аналог & Новый вариант &  \\ 
\hline Простота использования & По 10-бальной шкале & 2 & 9 & 0.05 \\ 
\hline Быстродействие & По 10-бальной шкале & 5 & 10 & 0.2 \\ 
\hline Открытость & По 10-бальной шкале & 3 & 10 & 0.15 \\ 
\hline Точность вычислений & По 10-бальной шкале & 8 & 8 & 0.2 \\ 
\hline Надёжность & По 10-бальной шкале & 9 & 7 & 0.2 \\ 
\hline 
\end{tabular}
%\end{tabularx} 
\end{table}
	\chapter{Охрана труда и окружающей среды}
\section{Анализ условий труда}
\subsection{Обеспечение условий труда в отделе разработки программного обеспечения}
Дипломная работа посвящена разработке системы мониторинга состояния ЛА на основе алгоритмов интеллектуального анализа данных. Разработка производится на персональном компьютере и предполагает длительное пребывание за ним инженера.

Применение персонального компьютера освобождает человека от непроизводительной работы, связанной с обработкой информации, изменяет характер его труда. Однако при этом увеличивается доля умственного и нервно-напряженного труда, возрастает психоэмоциональная нагрузка. При значительной трудовой нагрузке, нерациональной организации работы и неблагоприятных факторах производственной среды быстро снижается работоспособность операторов, уменьшается производительность труда и ухудшается качество работы, может развиться перенапряжение,~а в отдельных случаях возникнуть срыв трудовой деятельности~--- дистресс.

В данном разделе проводится анализ условий труда в отделе разработки информационных систем с целью обеспечения безопасности и удобства, требуемых для работы инженера.

\subsection{Характеристика помещения}
Помещение находится в здании Московского Авиационного Института и представляет собой кафедральную лабораторию со следующими параметрами:
\begin{itemize}
	\item длина~6~м;
	\item ширина~4~м;
	\item высота~3,5~м.
\end{itemize}

Общая площадь: $6\times4 = 24$~м\textsuperscript{2}.

Объём: $6\times4\times3,5 = 84$~м\textsuperscript{3}.

Количество рабочих мест~---~4.

Количество одновременно находящихся в помещении сотрудников не превышает~4 человек.

План помещения приведён на рисунке~\ref{fig:room_plan}.
\begin{figure}[h]
\includegraphics[width=0.6\textwidth, keepaspectratio]{plan}
\caption{План помещения}\label{fig:room_plan}
\end{figure}

Нормативные требования к площади и объёму рабочих мест определены в \cite{SanPin2_2_2}:
\begin{itemize}
	\item площадь на одно рабочее место с ВДТ или ПЭВМ для взрослых пользователей должна составлять не~менее~6~м\textsuperscript{2};
	\item объём~---~не~менее~20~м\textsuperscript{3}.
\end{itemize}

Фактические значения на каждого сотрудника:
\begin{itemize}
	\item площадь: $24/4 = 6$~м\textsuperscript{2};
	\item объём: $84/4 = 21$~м\textsuperscript{3}.
\end{itemize}

Данные значения показывают, что кафедральная лаборатория полностью соответствует установленным нормам.

В помещении имеются 2 оконных проёма высотой 1,6~м и шириной 2,3~м, которые выходят на юго-запад.

Искусственное освещение представляет собой 6~потолочных ламп, расположенных параллельно окнам в~2~ряда.

\subsection{Характеристика производственного процесса}
Разработка программного обеспечения производится на ПЭВМ с подключенными к ней периферийными устройствами.

\subsection{Характеристика используемого оборудования}
В процессе разработки используется следующее оборудование:
\begin{enumerate}
	\item{ПЭВМ:}
		\begin{enumerate}
			\item{процессор Intel Core i5 3,60 ГГц;}
			\item{оперативная память 8 Гб;}
			\item{жёсткий диск 1 Tб;}
			\item{напряжение питания 220 В.}
		\end{enumerate}
	\item{ЖК монитор с диагональю 23 дюйма (58,42) ASUS VX239H:}
		\begin{enumerate}
			\item{частота 75 Гц;}
			\item{яркость 250 кд/м\textsuperscript{2};}
			\item{динамическая контрастность 8 000 000 : 1;}
			\item{напряжение питания 220 В.}
		\end{enumerate}
	\item{Клавиатура Logitech K330;}
	\item{Мышь A7Tech X;}
	\item{Принтер HP LaserJet 1005M:}
		\begin{enumerate}
			\item{напряжение питания 220 В.}
		\end{enumerate}
\end{enumerate}

\subsection{Санитарно-гигиенические факторы}

\subsubsection{Микроклимат помещения}
Микроклимат в рабочем помещении должен соответствовать \cite{GOST12_1_005}.

Согласно \cite{GOST12_1_005}, работа разработчика ПО относится к категории «Легкая~–~Iа», т.к. лёгкие физические работы~--- работы с расходом энергии не более 150 ккал (174 Вт), а категория Iа подразумевает энергозатраты до 120~ккал/ч~(139 Вт).

Рабочее место разработчика ПО является постоянным, т.к. он находится на нём большую часть рабочего времени (более~50\%).

Нормативные и фактические значения для категории работ «Легкая~–~Iа» и постоянного рабочего места приведены в таблице~\ref{tab:microclimat}.

\begin{table}[h]
	\caption{Значения характеристик микроклимата помещения}\label{tab:microclimat}
	%\begin{tabular}{|p{2.5cm}|p{3.5cm}|p{3cm}|p{4cm}|}
	\begin{tabular}{|c|c|c|c|}
		\hline
		& \makecell{Температура,~\textdegree C} & \makecell{Относительная влажность,~\%} & \makecell{Скорость движения,~м/с} \\
		\hline
		\makecell{Допустимые значения} & \makecell{22--24 --- холодный период} & 40--60 & 0,1 \\
		\hline
		\makecell{Фактические значения} & \makecell{22--24 --- холодный период} & 44--55 & <0,1 \\
		\hline
	\end{tabular}
\end{table}

Согласно \cite{GOST12_1_005}, на рабочем месте значение относительной влажности и скорости движения воздуха удовлетворяют оптимальным значениям. Фактические значения характеристик микроклимата данного помещения удовлетворяют допустимым значениям для холодного и теплого периода.
	\include{conclusion}
	
	\addcontentsline{toc}{chapter}{\bibname}
	\bibliography{thesis}
	
	\appendix
%	\appendixtitleon
%	\appendixtitletocon
	\begin{appendices}
	\chapter{Исходный код}
*тут должен быть исходный код*
	\include{appendix2}
	\end{appendices}
\end{document}