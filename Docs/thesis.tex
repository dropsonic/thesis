\documentclass[oneside,final,14pt]{extarticle}
\usepackage[onehalfspacing]{setspace} % 1,5 интервал
\usepackage[top=2.0cm,bottom=2.0cm,left=2.0cm,right=1.0cm]{geometry} % поля
\usepackage{lscape} % поддержка альбомной ориентации страниц
\usepackage{indentfirst} % красная строка

\usepackage[cm-default]{fontspec}
\usepackage{xunicode}
\usepackage{xltxtra}
%\setromanfont{Cambria}
\setromanfont{Times New Roman}
\setsansfont{Calibri}
\setmonofont{Consolas}
\defaultfontfeatures{Scale=MatchLowercase} % одинаковый рост строчных букв у разных гарнитур
\usepackage{ulem} % поддержка подчёркиваний

% Поддержка русского языка и русскоязычных стилей
\usepackage{polyglossia}
\setmainlanguage[babelshorthands=true]{russian} % основной язык - русский
\setotherlanguage[variant=us]{english} % дополнительный язык - английский

\usepackage{graphicx} % поддержка изображений
\graphicspath{{images/}} % путь к каталогу с изображениями

\usepackage{amstext} % поддержка кириллицы в математических формулах
\usepackage{amssymb} % дополнительные символы в математических формулах


% Красивые таблицы
\usepackage{makecell}
\usepackage{multirow}

\begin{document}
\section*{Мой диплом в \XeLaTeX}
Мой диплом на 90\% состоит из воды. Значит, мой диплом~— человек.
\subsection*{Кафедра 308}
В настоящее время кафедра осуществляет подготовку по двухуровневой системе «бакалавр-магистр» по образовательным программам направления 230400 «Информационные системы и технологии» (бакалавр) профили «Информационные системы аэрокосмических комплексов», «Информационные системы испытаний космических летательных аппаратов» и магистров направления 230100 «Вычислительная техника и информатика» по программе «Программное и алгоритмическое обеспечение систем идентификации объектов по их изображениям». Нормативный срок обучения по программе бакалавриата — 4 года, а по программе магистратуры — 2 года. Обучение студентов осуществляется как на госбюджетной (бесплатной), так и на контрактной (платной) основах. 
\end{document}