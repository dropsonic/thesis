\chapter{Расчет экономической эффективности системы}
\section{Введение}
Для оценки экономической эффективности программно-аппаратного продукта требуется:
\begin{itemize}
\item определить целесообразность разработки;
\item определить трудоёмкость и затраты на создание;
\item определить показатели экономической эффективности разработки.
\end{itemize}

Результатом выполнения данной части является обоснование технической, экономической и научной значимости и целесообразности продукта. Объектом технико-экономического анализа является программная система мониторинга состояния ЛА на основе методов интеллектуального анализа данных.

\section{Определение целесообразности разработки}
Для обоснования целесообразности разработки продукта необходимо:
\begin{itemize}
\item выбрать аналог (если таковой имеется);
\item сформулировать перечень функциональных характеристик по предлагаемому варианту разработки продукта;
\item определить конкретные уровни характеристик и их значимость;
\item определить индекс технического уровня программного продукта.
\end{itemize}

Функционально-технические характеристики разрабатываемого программного продукта представлены в таблице~\ref{tab:econom:characteristics}.

\begin{table}
\caption{Функционально-технические характеристики}
\nohyphenation
\label{tab:econom:characteristics}

\begin{tabular}{|C{120.05pt}|C{77.95pt}|C{72pt}|C{72pt}|C{108pt}|}
\hline
\multirow{2}{\hsize}{\centering{Функциональные характеристики}} & \multirow{2}{\hsize}{\centering{Единица измерения}} & \multicolumn{2}{C{144pt}|}{Величина функциональных характеристик} & \multirow{2}{\hsize}{\centering{Значимость характеристик}} \\
\cline{3-4}
 & & Аналог & Новый вариант & \\
\hline
Простота использования & По 10-бальной шкале & 2 & 9 & 0.05 \\
\hline
Быстродействие & По 10-бальной шкале & 5 & 10 & 0.2 \\
\hline
Открытость & По 10-бальной шкале & 3 & 10 & 0.15 \\
\hline
Точность вычислений & По 10-бальной шкале & 8 & 8 & 0.2 \\
\hline
Надёжность & По 10-бальной шкале & 9 & 7 & 0.2 \\
\hline
\end{tabular}
\end{table}