\chapter{Охрана труда и окружающей среды}
\section{Анализ условий труда}
\subsection{Обеспечение условий труда в отделе разработки программного обеспечения}
Дипломная работа посвящена разработке системы мониторинга состояния ЛА на основе алгоритмов интеллектуального анализа данных. Разработка производится на персональном компьютере и предполагает длительное пребывание за ним инженера.

Применение персонального компьютера освобождает человека от непроизводительной работы, связанной с обработкой информации, изменяет характер его труда. Однако при этом увеличивается доля умственного и нервно-напряженного труда, возрастает психоэмоциональная нагрузка. При значительной трудовой нагрузке, нерациональной организации работы и неблагоприятных факторах производственной среды быстро снижается работоспособность операторов, уменьшается производительность труда и ухудшается качество работы, может развиться перенапряжение,~а в отдельных случаях возникнуть срыв трудовой деятельности~— дистресс.

\subsection{Характеристика помещения}
Помещение находится в здании Московского Авиационного Института и представляет собой кафедральную лабораторию со следующими параметрами:
\begin{itemize}
\item длина 6 м;
\item ширина 4 м;
\item высота 3,5 м.
\end{itemize}

Общая площадь: 6×4 = 24 м2.
Объём: 6×4×3,5 = 84 м3.
Количество рабочих мест~–~4.
Количество одновременно находящихся в помещении сотрудников не превышает~4 человек.