\chapter{Охрана труда и окружающей среды}
\section{Анализ условий труда}
\subsection{Обеспечение условий труда в отделе разработки программного обеспечения}
Дипломная работа посвящена разработке системы мониторинга состояния ЛА на основе алгоритмов интеллектуального анализа данных. Разработка производится на персональном компьютере и предполагает длительное пребывание за ним инженера.

Применение персонального компьютера освобождает человека от непроизводительной работы, связанной с обработкой информации, изменяет характер его труда. Однако при этом увеличивается доля умственного и нервно-напряженного труда, возрастает психоэмоциональная нагрузка. При значительной трудовой нагрузке, нерациональной организации работы и неблагоприятных факторах производственной среды быстро снижается работоспособность операторов, уменьшается производительность труда и ухудшается качество работы, может развиться перенапряжение,~а в отдельных случаях возникнуть срыв трудовой деятельности~--- дистресс.

В данном разделе проводится анализ условий труда в отделе разработки информационных систем с целью обеспечения безопасности и удобства, требуемых для работы инженера.

\subsection{Характеристика помещения}
Помещение находится в здании Московского Авиационного Института и представляет собой кафедральную лабораторию со следующими параметрами:
\begin{itemize}
	\item длина~6~м;
	\item ширина~4~м;
	\item высота~3,5~м.
\end{itemize}

Общая площадь: $6\times4 = 24$~м\textsuperscript{2}.

Объём: $6\times4\times3,5 = 84$~м\textsuperscript{3}.

Количество рабочих мест~---~4.

Количество одновременно находящихся в помещении сотрудников не превышает~4 человек.

План помещения приведён на рисунке~\ref{image:room_plan}.
\begin{figure}[h]
\includegraphics[width=0.6\textwidth, keepaspectratio]{plan}
\caption{План помещения}\label{image:room_plan}
\end{figure}

\begin{table}[h]
\caption{\label{tab:canonsummary}Измерительные характеристики цифровой камеры Canon EOS 400D}
\begin{tabular}{|c|c|}
\hline
Параметр & Значение \\
\hline
Разрешение & $3888 \times 2592$ \\
Размер сенсора & $22.2 \times 14.8$ мм \\
АЦП & 12~bit\\
\hline
\multicolumn{2}{|c|}{Результаты измерений} \\
\hline
Темновое смещение (BLO) & 256 \\
Максимальный линейный сигнал & 3070~DN \\
Значение насыщения & 3470~DN \\
\hline
\end{tabular}
\end{table} 

\begin{equation} 
f(x,y,\alpha, \beta) = \frac{\sum \limits_{n=1}^{\infty} 
A_n \cos \left( \frac{2 n \pi x}{\nu} \right)} {\prod \mathcal{F} {g(x,y)} } 
\end{equation}